\documentclass{wissdoc}
% Autor: Roland Bless 1996-2009, bless <at> kit.edu
% ----------------------------------------------------------------
% Diplomarbeit - Hauptdokument
% ----------------------------------------------------------------
%%
%% $Id: thesis.tex 65 2012-05-10 10:32:11Z bless $
%%
%
% Zum Erstellen zweiseitiger PDFs (für Buchdruck) in der Datei "wissdoc.cls" folgende Zeile abändern:
%
% \LoadClass[a4paper,12pt,oneside]{book} % diese Klasse basiert auf ``book''
% in
%\LoadClass[a4paper,12pt,titlepage]{book} % diese Klasse basiert auf ``book''
%
%
% wissdoc Optionen: draft, relaxed, pdf --> siehe wissdoc.cls
% ------------------------------------------------------------------
% Weitere packages: (Dokumentation dazu durch "latex <package>.dtx")
\usepackage[numbers,sort&compress]{natbib}
% \usepackage{varioref}
% \usepackage{verbatim}
% \usepackage{float}    %z.B. \floatstyle{ruled}\restylefloat{figure}
% \usepackage{subfigure}
% \usepackage{fancybox} % für schattierte,ovale Boxen etc.
% \usepackage{tabularx} % automatische Spaltenbreite
% \usepackage{supertab} % mehrseitige Tabellen
% \usepackage[svnon,svnfoot]{svnver} % SVN Versionsinformation

% \usepackage[ngerman]{babel} <-- German only

%% ---------------- end of usepackages -------------

%\svnversion{$Id: thesis.tex 65 2012-05-10 10:32:11Z bless $} % In case that you want to include version information in the footer

%% Informationen für die PDF-Datei
\hypersetup{
 pdfauthor={N.N.},
 pdftitle={Not set}
 pdfsubject={Not set},
 pdfkeywords={Not set}
}

% Macros, nicht unbedingt notwendig
%%%%%%%%%%%%%%%%%%%%%%%%%%%%%%%%%%%%%%%%%%%%%%%%%%%%%%%%%%
% macros.tex -- einige mehr oder weniger nuetzliche Makros
% Autor: Roland Bless 1998
%%%%%%%%%%%%%%%%%%%%%%%%%%%%%%%%%%%%%%%%%%%%%%%%%%%%%%%%%%
% $Id: macros.tex 33 2007-01-23 09:00:59Z bless $
%%%%%%%%%%%%%%%%%%%%%%%%%%%%%%%%%%%%%%%%%%%%%%%%%%%%%%%%%%


%%%%%%%%%%%%%%%%%%%%%%%
% Kommentare 
%%%%%%%%%%%%%%%%%%%%%%%
\ifnotdraftelse{
\newcommand{\Kommentar}[1]{}
}{\newcommand{\Kommentar}[1]{{\em #1}}}
% Alles innerhalb von \Hide{} oder \ignore{} 
% wird von LaTeX komplett ignoriert (wie ein Kommentar)
\newcommand{\Hide}[1]{}
\let\ignore\Hide

%%%%%%%%%%%%%%%%%%%%%%%%%
% Leere Seite ohne Seitennummer, wird aber gezaehlt
%%%%%%%%%%%%%%%%%%%%%%%%%

\newcommand{\leereseite}{% Leerseite ohne Seitennummer, nächste Seite rechts (wenn 2-seitig)
 \clearpage{\pagestyle{empty}\cleardoublepage}
}
%%%%%%%%%%%%%%%%%%%%%%%%%%
% Flattersatz rechts und Silbentrennung, Leerraum nach rechts maximal 1cm
%%%%%%%%%%%%%%%%%%%%%%%%%%
\makeatletter
\newcommand{\myraggedright}{%
 \let\\\@centercr\@rightskip 0pt plus 1cm
 \rightskip\@rightskip
  \leftskip\z@skip
  \parindent\z@
  \spaceskip=.3333em
  \xspaceskip=.5em}
\makeatother

\makeatletter
\newcommand{\mynewline}{%
 \@centercr\@rightskip 0pt plus 1cm
}
\makeatother


%%%%%%%%%%%%%%%%%%%%%%%%%%
% Für Index
%%%%%%%%%%%%%%%%%%%%%%%%%%
\makeatletter
\def\mydotfill{\leavevmode\xleaders\hb@xt@ .44em{\hss.\hss}\hfill\kern\z@}
\makeatother
\def\bold#1{{\bfseries #1}}
\newbox\dbox \setbox\dbox=\hbox to .4em{\hss.\hss} % dot box for leaders
\newskip\rrskipb \rrskipb=.5em plus3em % ragged right space before break
\newskip\rrskipa \rrskipa=-.17em plus -3em minus.11em % ditto, after
\newskip\rlskipa \rlskipa=0pt plus3em % ragged left space after break
\newskip\rlskipb \rlskipb=.33em plus-3em minus.11em % ragged left before break
\newskip\lskip \lskip=3.3\wd\dbox plus1fil minus.3\wd\dbox % for leaders
\newskip \lskipa \lskipa=-2.67em plus -3em minus.11em %after leaders
\mathchardef\rlpen=1000 \mathchardef\leadpen=600
\def\rrspace{\nobreak\hskip\rrskipb\penalty0\hskip\rrskipa}
\def\rlspace{\penalty\rlpen\hskip\rlskipb\vadjust{}\nobreak\hskip\rlskipa}
\let\indexbreak\rlspace
\def\raggedurl{\penalty10000 \hskip.5em plus15em \penalty0 \hskip-.17em plus-15em minus.11em}
\def\raggeditems{\nobreak\hskip\rrskipb \penalty\leadpen \hskip\rrskipa %
\vadjust{}\nobreak\leaders\copy\dbox\hskip\lskip %
\kern3em \penalty\leadpen \hskip\lskipa %
\vadjust{}\nobreak\hskip\rlskipa}
%\renewcommand*\see[2]{\rlspace\emph{\seename}~#1} % from makeidx.sty

%%%%%%%%%%%%%%%%%%%%%%%%%%
% Neue Seite rechts, leere linke Seite ohne Headings
%%%%%%%%%%%%%%%%%%%%%%%%%%
\newcommand{\xcleardoublepage}
{{\pagestyle{empty}\cleardoublepage}}

%%%%%%%%%%%%%%%%%%%%%%%%%%
% Tabellenspaltentypen (benoetigt colortbl)
%%%%%%%%%%%%%%%%%%%%%%%%%%
\newcommand{\PBS}[1]{\let\temp=\\#1\let\\=\temp}
\newcolumntype{y}{>{\PBS{\raggedright\hspace{0pt}}}p{1.35cm}}
\newcolumntype{z}{>{\PBS{\raggedright\hspace{0pt}}}p{2.5cm}}
\newcolumntype{q}{>{\PBS{\raggedright\hspace{0pt}}}p{6.5cm}}
\newcolumntype{g}{>{\columncolor[gray]{0.8}}c} % Grau
\newcolumntype{G}{>{\columncolor[gray]{0.9}}c} % helleres Grau

%%%%%%%%%%%%%%%%%%%%%%%%%%
% Anführungszeichen oben und unten
%%%%%%%%%%%%%%%%%%%%%%%%%%
\newcommand{\anf}[1]{"`{#1}"'}

%%%%%%%%%%%%%%%%%%%%%%%%%%
% Tiefstellen von Text
%%%%%%%%%%%%%%%%%%%%%%%%%%
% S\tl{0} setzt die 0 unter das S (ohne Mathemodus!)
% zum Hochstellen gibt es uebrigens \textsuperscript
\makeatletter
\DeclareRobustCommand*\textlowerscript[1]{%
  \@textlowerscript{\selectfont#1}}
\def\@textlowerscript#1{%
  {\m@th\ensuremath{_{\mbox{\fontsize\sf@size\z@#1}}}}}
\let\tl\textlowerscript
\let\ts\textsuperscript
\makeatother

%%%%%%%%%%%%%%%%%%%%%%%%%%
% Gauß-Klammern
%%%%%%%%%%%%%%%%%%%%%%%%%%
\newcommand{\ceil}[1]{\lceil{#1}\rceil}
\newcommand{\floor}[1]{\lfloor{#1}\rfloor}

%%%%%%%%%%%%%%%%%%%%%%%%%%
% Average Operator (analog zu min, max)
%%%%%%%%%%%%%%%%%%%%%%%%%%
\def\avg{\mathop{\mathgroup\symoperators avg}}

%%%%%%%%%%%%%%%%%%%%%%%%%%
% Wortabkürzungen
%%%%%%%%%%%%%%%%%%%%%%%%%%
\def\zB{z.\,B.\ }
\def\dh{d.\,h.\ }
\def\ua{u.\,a.\ }
\def\su{s.\,u.\ }
\newcommand{\bzw}{bzw.\ }

%%%%%%%%%%%%%%%%%%%%%%%%%%%%%%%%%%%
% Einbinden von Graphiken
%%%%%%%%%%%%%%%%%%%%%%%%%%%%%%%%%%%
% global scaling factor
\def\gsf{0.9}
%% Graphik, 
%% 3 Argumente: Datei, Label, Unterschrift
\newcommand{\Abbildung}[3]{%
\begin{figure}[tbh] %
\centerline{\scalebox{\gsf}{\includegraphics*{#1}}} %
\caption{#3} %
\label{#2} %
\end{figure} %
}
\let\Abb\Abbildung
%% Abbps
%% Graphik, skaliert, Angabe der Position
%% 5 Argumente: Position, Breite (0 bis 1.0), Datei, Label, Unterschrift
\newcommand{\Abbildungps}[5]{%
\begin{figure}[#1]%
\begin{center}
\scalebox{\gsf}{\includegraphics*[width=#2\textwidth]{#3}}%
\caption{#5}%
\label{#4}%
\end{center}
\end{figure}%
}
\let\Abbps\Abbildungps
%% Graphik, Angabe der Position, frei wählbares Argument für includegraphics
%% 5 Argumente: Position, Optionen, Datei, Label, Unterschrift
\newcommand{\Abbildungpf}[5]{%
\begin{figure}[#1]%
\begin{center}
\scalebox{\gsf}{\includegraphics*[#2]{#3}}%
\caption{#5}%
\label{#4}%
\end{center}
\end{figure}%
}
\let\Abbpf\Abbildungpf

%%
% Anmerkung: \resizebox{x}{y}{box} skaliert die box auf Breite x und Höhe y,
%            ist x oder y ein !, dann wird das usprüngliche 
%            Seitenverhältnis beibehalten.
%            \rescalebox funktioniert ähnlich, nur das dort ein Faktor
%            statt einer Dimension angegeben wird.
%%
% \Abbps{Position}{Breite in Bruchteilen der Textbreite}{Dateiname}{Label}{Bildunterschrift}
%

\newcommand{\refAbb}[1]{%
s.~Abbildung \ref{#1}}

%%%%%%%%%%%%%%%%%%%%
%% end of macros.tex
%%%%%%%%%%%%%%%%%%%%

% Print URLs not in Typewriter Font
\def\UrlFont{\rm}

\newcommand{\blankpage}{% Leerseite ohne Seitennummer, nächste Seite rechts
 \clearpage{\pagestyle{empty}\cleardoublepage}
}

%% Einstellungen für das gesamte Dokument

% Trennhilfen
% Wichtig!
% Im ngerman-paket sind zusätzlich folgende Trennhinweise enthalten:
% "- = zusätzliche Trennstelle
% "| = Vermeidung von Ligaturen und mögliche Trennung (bsp: Schaf"|fell)
% "~ = Bindestrich an dem keine Trennung erlaubt ist (bsp: bergauf und "~ab)
% "= = Bindestrich bei dem Worte vor und dahinter getrennt werden dürfen
% "" = Trennstelle ohne Erzeugung eines Trennstrichs (bsp: und/""oder)

% Trennhinweise fuer Woerter hier beschreiben
\hyphenation{
% Pro-to-koll-in-stan-zen
}

% Index-Datei öffnen
\ifnotdraft{\makeindex}

\begin{document}

\frontmatter
\pagenumbering{roman}
\ifnotdraft{
 %% Titelseite
%% Vorlage $Id: titelseite.tex 61 2012-05-03 13:58:03Z bless $

\def\usesf{}
\let\usesf\sffamily % diese Zeile auskommentieren für normalen TeX Font

\newsavebox{\Erstgutachter}
\savebox{\Erstgutachter}{\usesf Prof.~Dr-Ing.~Ralf ~Schenkel}
\newsavebox{\Zweitgutachter}
\savebox{\Zweitgutachter}{\usesf <Name of second examiner>}
\newsavebox{\Betreuer}
\savebox{\Betreuer}{\usesf <Name of supervisor>}

\begin{titlepage}
\setlength{\unitlength}{1pt}
\begin{picture}(0,0)(85,770)
%\includegraphics[width=\paperwidth]{logos/KIT_Deckblatt}
\end{picture}

\thispagestyle{empty}

%\begin{titlepage}
%%\let\footnotesize\small \let\footnoterule\relax
\begin{flushright}
	\includegraphics[width=0.75\textwidth]{img/logo-trier.png}
\end{flushright}
\begin{center}
\hbox{}
\vfill
{\usesf
{\huge\bfseries Title of the thesis \par}
\vskip 1.8cm
{\huge Bachelor Thesis / Master's Thesis}\\
\vskip 0.5cm
for obtaining the academic degree\\
Bachelor/Master of Science (B.Sc./M.Sc.) 
\vskip 1.5cm

{\large Universität Trier\\
FB IV - Informatikwissenschaften\\
Professur für Datenbanken und Informationssysteme\\}

\vskip 2cm
\begin{tabular}{p{3.5cm}l}
Reviewer: & \usebox{\Erstgutachter} \\
 & \usebox{\Zweitgutachter} \\
Supervisor: & \usebox{\Betreuer} \\
\end{tabular}
\vskip 2cm
submitted on xx.xx.xxxx by:\\
\vskip .3cm
First and last name\\
street etc.\\
ZIP code and city\\
email address\\
Matr.-Nr. 123456


}
\end{center}
\vfill
\end{titlepage}
%% Titelseite Ende


%%% Local Variables: 
%%% mode: latex
%%% TeX-master: "thesis"
%%% End: 

 %\blankpage % Leerseite auf Titelrückseite
 \chapter*{Abstract}
%% ==============================
Here is a brief summary (abstract) of the work. Present briefly and precisely the aim and subject of the work, the methods used, as well as the results of the work. Keep the first points rather short and focus on the results. Also evaluate the results and place them in context.

The short summary should be about half a page to a full page long.

}
%
%% *************** Hier geht's ab ****************
%% ++++++++++++++++++++++++++++++++++++++++++
%% Verzeichnisse
%% ++++++++++++++++++++++++++++++++++++++++++
\ifnotdraft{
{\parskip 0pt\tableofcontents} % toc bitte einzeilig
%\blankpage
\listoffigures
%\blankpage
\listoftables
%\blankpage
}


%% ++++++++++++++++++++++++++++++++++++++++++
%% main part
%% ++++++++++++++++++++++++++++++++++++++++++
\graphicspath{{img/}}

\mainmatter
\pagenumbering{arabic}
%% Einleitung.tex
%% $Id: einleitung.tex 61 2012-05-03 13:58:03Z bless $
%%

\chapter{Introduction}
\label{ch:intro}
%% ==============================
The introduction consists of the motivation, the problem, the objective and a first overview of the structure of the paper.

%% ==============================
\section{Motivation}
%% ==============================
\label{ch:intro:sec:Motivation}

Why is the topic area to be worked on exciting and relevant?

%% ==============================
\section{Problem}
%% ==============================
\label{ch:intro:sec:problem}

What problem(s) can be identified in this topic area?

%% ==============================
\section{Objective}
%% ==============================
\label{ch:intro:sec:objective}

What is the goal of the work. How should the problem be solved?


%% ==============================
\section{Structure of the thesis}
%% ==============================
\label{ch:intro:sec:structure}

What do the other chapters contain? How is the work structured? What methodology is followed?


%%% Local Variables: 
%%% mode: latex
%%% TeX-master: "thesis"
%%% End: 
  % Introduction, Motivation
%% grundlagen.tex
%% $Id: grundlagen.tex 61 2012-05-03 13:58:03Z bless $
%%

\chapter{Foundations}
\label{ch:foundations}
%% ==============================
The foundations must be described to the extent that a reader can understand the problem and the solution to the problem without consulting further literature. It is important that the foundations are not detached from the rest of the paper. Here should also be a very light introduction to the topic including an explanation of all important basics (terms, techniques, algorithms, etc.).


%% ==============================
\section{Topic 1}
%% ==============================
\label{ch:foundations:sec:section1}

...

%% ==============================
\section{Topic 2}
%% ==============================
\label{ch:foundations:sec:section2}

...

%% ==============================
\section{State of the Art}
%% ==============================
\label{ch:foundations:sec:SOTA}
The literature search should be as complete as possible and should describe or briefly present already existing relevant approaches (related work / state of the art / state of the art). It should be shown where these approaches have deficits or are not applicable, e.g. because they assume different environments or prerequisites.

Depending on the type of thesis, it may also make sense to integrate this section into the introduction or to list it as a separate chapter.

Example of how to cite with LaTeX: \cite{TB98,JSAC96,qosr}

%%% Local Variables: 
%%% mode: latex
%%% TeX-master: "thesis"
%%% End: 
  % Foundations, including related work
%% analyse.tex
%% $Id: analyse.tex 61 2012-05-03 13:58:03Z bless $

\chapter{Problem and Analysis}
\label{ch:analysis}
%% ==============================
In this chapter, the problem to be solved as well as the requirements and the boundary conditions of a solution shall be described first (a more precise problem definition).


%% ==============================
\section{Requirements}
%% ==============================
\label{ch:analysis:sec:requirements}
Requirements and limiting factors

%% ==============================
\section{Additional section}
%% ==============================
\label{ch:analysis:sec:section}

Lorem ipsum dolor sit amet, consetetur sadipscing elitr, sed diam nonumy eirmod tempor invidunt ut labore et dolore magna aliquyam erat, sed diam voluptua. At vero eos et accusam et justo duo dolores et ea rebum. Stet clita kasd gubergren, no sea takimata sanctus est Lorem ipsum dolor sit amet.

\begin{figure}[htb]
\centering
  	{\includegraphics[width=.3\textwidth]{Logo-Uni-Trier.jpg}}
	\caption{Logo of Trier University.\label{fig:grafik1}}
\centering
\end{figure}

Lorem ipsum dolor sit amet, consetetur sadipscing elitr, sed diam nonumy eirmod tempor invidunt ut labore et dolore magna aliquyam erat, sed diam voluptua. 

\begin{table}[htb]
\caption{Table with values.\label{tab:list}}
\vspace*{1em}
\centering

\bgroup
\def\arraystretch{1.3}%  1 is the default, change whatever you need

\begin{tabular}[c]{l|l|c}
	
	\multicolumn{1}{c|}{\textbf{A}} & 
	\multicolumn{1}{c|}{\textbf{B}} & 
	\multicolumn{1}{c}{\textbf{C}} \\ 
	
	\hline

	Test 1& Slow& 279 \\ 
	&Fast & 499 \\ 
	&Very fast& 719 \\ 
	
\end{tabular}

\egroup

\end{table}

Duis autem vel eum iriure dolor in hendrerit in vulputate velit esse molestie consequat, vel illum dolore eu feugiat nulla facilisis at vero eros et accumsan et iusto odio dignissim qui blandit praesent luptatum zzril delenit augue duis dolore te feugait nulla facilisi. 

Duis autem vel eum iriure dolor in hendrerit in vulputate velit esse molestie consequat, vel illum dolore eu feugiat nulla facilisis. 

%% ==============================
\section{Summary}
%% ==============================
\label{ch:analysis:sec:summary}

At the end, if necessary, the most important results should be summarized again in \emph{a}
short paragraph.

%%% Local Variables: 
%%% mode: latex
%%% TeX-master: "thesis"
%%% End: 
     % Analysis
%% entwurf.tex
%% $Id: entwurf.tex 61 2012-05-03 13:58:03Z bless $
%%

\chapter{Design / Conception}
\label{ch:design}
%% ==============================
This chapter contains a detailed description of our own solution approach.
solution approach. Alternative solutions should be discussed and
design decisions should be presented.

%% ==============================
\section{Section 1}
%% ==============================
\label{ch:design:sec:section1}

Lorem ipsum dolor sit amet, consetetur sadipscing elitr, sed diam nonumy eirmod tempor invidunt ut labore et dolore magna aliquyam erat, sed diam voluptua. At vero eos et accusam et justo duo dolores et ea rebum. Stet clita kasd gubergren, no sea takimata sanctus est Lorem ipsum dolor sit amet. Lorem ipsum dolor sit amet, consetetur sadipscing elitr, sed diam nonumy eirmod tempor invidunt ut labore et dolore magna aliquyam erat, sed diam voluptua. At vero eos et accusam et justo duo dolores et ea rebum. Stet clita kasd gubergren, no sea takimata sanctus est Lorem ipsum dolor sit amet. Lorem ipsum dolor sit amet, consetetur sadipscing elitr, sed diam nonumy eirmod tempor invidunt ut labore et dolore magna aliquyam erat, sed diam voluptua. At vero eos et accusam et justo duo dolores et ea rebum. Stet clita kasd gubergren, no sea takimata sanctus est Lorem ipsum dolor sit amet. 

%% ==============================
\section{Section 2}
%% ==============================
\label{ch:design:sec:section2}

Lorem ipsum dolor sit amet, consetetur sadipscing elitr, sed diam nonumy eirmod tempor invidunt ut labore et dolore magna aliquyam erat, sed diam voluptua. At vero eos et accusam et justo duo dolores et ea rebum. Stet clita kasd gubergren, no sea takimata sanctus est Lorem ipsum dolor sit amet. Lorem ipsum dolor sit amet, consetetur sadipscing elitr, sed diam nonumy eirmod tempor invidunt ut labore et dolore magna aliquyam erat, sed diam voluptua. At vero eos et accusam et justo duo dolores et ea rebum. Stet clita kasd gubergren, no sea takimata sanctus est Lorem ipsum dolor sit amet. Lorem ipsum dolor sit amet, consetetur sadipscing elitr, sed diam nonumy eirmod tempor invidunt ut labore et dolore magna aliquyam erat, sed diam voluptua. At vero eos et accusam et justo duo dolores et ea rebum. Stet clita kasd gubergren, no sea takimata sanctus est Lorem ipsum dolor sit amet. 

Duis autem vel eum iriure dolor in hendrerit in vulputate velit esse molestie consequat, vel illum dolore eu feugiat nulla facilisis at vero eros et accumsan et iusto odio dignissim qui blandit praesent luptatum zzril delenit augue duis dolore te feugait nulla facilisi. Lorem ipsum dolor sit amet, consectetuer adipiscing elit, sed diam nonummy nibh euismod tincidunt ut laoreet dolore magna aliquam erat volutpat. 

%% ==============================
\section{Summary}
%% ==============================
\label{ch:design:sec:summary}

At the end, if necessary, the most important results should be summarized again in \emph{a}
short paragraph.

%%% Local Variables: 
%%% mode: latex
%%% TeX-master: "thesis"
%%% End: 
     % design
%% implemen.tex
%% $Id: implemen.tex 61 2012-05-03 13:58:03Z bless $
%%

\chapter{Implementation}
\label{ch:implementation}
%% ==============================
\ldots

%% ==============================
\section{Section 1}
%% ==============================
\label{ch:implementation:sec:section1}

\ldots

%% ==============================
\section{Section 2}
%% ==============================
\label{ch:implementation:sec:section2}

\ldots

%%% Local Variables: 
%%% mode: latex
%%% TeX-master: "thesis"
%%% End: 
    % Implementation
%% eval.tex
%% $Id: eval.tex 61 2012-05-03 13:58:03Z bless $

\chapter{Evaluation}
\label{ch:evaluation}
%% ==============================
Here is the proof that the concept designed in chapter~\ref{ch:design}
works. 
Performance measurements of an implementation are always welcome.

%% ==============================
\section{Section 1}
%% ==============================
\label{ch:evaluation:sec:Section1}

\ldots

%% ==============================
\section{Section 2}
%% ==============================
\label{ch:evaluation:sec:Section2}

\ldots

%% ==============================
\section{Summary}
%% ==============================
\label{ch:evaluation:sec:summary}

At the end, if necessary, the most important results should be summarized again in \emph{a}
short paragraph.

%%% Local Variables: 
%%% mode: latex
%%% TeX-master: "thesis"
%%% End: 
        % Evaluation
%% zusammenf.tex
%% $Id: zusammenf.tex 61 2012-05-03 13:58:03Z bless $
%%

\chapter{Discussion und future work}
\label{ch:conclusion}
%% ==============================

(no sections here)

%%% Local Variables: 
%%% mode: latex
%%% TeX-master: "thesis"
%%% End: 
   	  % discussion and future work

%% ++++++++++++++++++++++++++++++++++++++++++
%% Appendix
%% ++++++++++++++++++++++++++++++++++++++++++

\appendix
%\include{appendix_a}
%\include{appendix_b}

%% ++++++++++++++++++++++++++++++++++++++++++
%% references
%% ++++++++++++++++++++++++++++++++++++++++++
%  mit dem Befehl \nocite werden auch nicht
%  zitierte Referenzen abgedruckt

\cleardoublepage
\phantomsection
\addcontentsline{toc}{chapter}{\bibname}
%%
\bibliographystyle{itmabbrv} % mit abgekürzten Vornamen der Autoren
%\bibliographystyle{gerplain} % abbrvnat unsrtnat
% spezielle Zitierstile: Labels mit vier Buchstaben und Jahreszahl
%\bibliographystyle{itmalpha}  % ausgeschriebene Vornamen der Autoren
\bibliography{thesis}

%% ++++++++++++++++++++++++++++++++++++++++++
%% Index
%% ++++++++++++++++++++++++++++++++++++++++++
%\ifnotdraft{
%\cleardoublepage
%\phantomsection
%\printindex            % Index, Stichwortverzeichnis
%}

 %
 % Die folgende Erklärung ist für Abschlussarbeiten Pflicht
 % (siehe Prüfungsordnung), für Studienarbeiten nicht notwendig
 \thispagestyle{empty}
%\vspace*{35\baselineskip}
%\hbox to \textwidth{\hrulefill}
\par

%% the regulations seem to require this statement in German.

\chapter*{Eidesstattliche Erklärung}

Hiermit erkläre ich, dass ich diese Bachelor-/Masterarbeit selbständig verfasst und keine anderen als die angegebenen Quellen und Hilfsmittel benutzt und die aus fremden Quellen direkt oder indirekt übernommenen Gedanken als solche kenntlich gemacht habe. Die Arbeit habe ich bisher keinem anderen Prüfungsamt in gleicher oder vergleichbarer Form vor-gelegt. Sie wurde bisher auch nicht veröffentlicht.

Trier, den xx. Monat 20xx

%%%%%%%%%%%%%%%%%%%%%%%%%%%%%%%%%%%%%%%%%%%%%%%%%%%%%%%%%%%%%%%%%%%%%%%%
%% Hinweis:
%%
%% Diese Erklärung wird von der Prüfungsordnung für Diplomarbeiten 
%% verlangt und ist zu unterschreiben. Für Studienarbeiten ist diese
%% Erklärung nicht zwingend notwendig, schadet aber auch nicht.
%%%%%%%%%%%%%%%%%%%%%%%%%%%%%%%%%%%%%%%%%%%%%%%%%%%%%%%%%%%%%%%%%%%%%%%%
\clearpage







 \blankpage % Leerseite auf Erklärungsrückseite

\end{document}
%% end of file
